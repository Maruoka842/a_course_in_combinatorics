\subsection{}
帰納法.$n$ について示されたとして,$n+2$ の場合に示す.
$G$ を $n+2$ 頂点かつ辺が $\lfloor \frac{(n+2)^2}{4}\rfloor$ 本とする.隣接する $2$ 点 $a,b$ をとり,$G_1 = G\setminus \{a,b\}$ とする.

$G_1$ は三角形を含まないので,$|E(G_1)|\leq \lfloor\frac{n^2}{4}\rfloor$.
また三角形の非存在より,各 $v\in V(G_1)$ は $a,b$ のうち高々一方としか隣接していない.よって $G_1$ と $\{a,b\}$ の間にある辺の本数は $|V(G_1)| = n$ 本以下.
したがって,$|E(G)|\leq \lfloor\frac{n^2}{4}\rfloor + n + 1$ が成り立つ.

$|E(G)|$ に対する仮定よりこの不等式において等号が成り立つ.よって議論に用いた不等式評価は全て最善であるから,
$|E(G_1)| = \lfloor\frac{n^2}{4}\rfloor$ であり,各 $v\in V(G_1)$ は$a,b$ のうちちょうど一方と隣接する.
このことと帰納法の仮定より,まず $G_1$ が $K_{k,k}$ あるいは $K_{k,k+1}$ であることが分かり,三角形ができないという条件から $v$ と $a,b$ の結び方も決まって $G$ に対する主張が従う.