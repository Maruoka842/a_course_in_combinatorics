\subsection{}
頂点集合を$\{A_i\}$とし、辺集合を$\{(u,v)\mid |u \Delta v|=1\}$ とするグラフ $G$ を考える($\Delta$ は対称差)。
さらに各辺に対し $f:E(G)\to N: (u,v) \mapsto u\Delta v$ により色を定める($N$ の1元集合と $N$ の要素を同一視した)。
定義より、$u\setminus \{x\}=v\setminus \{x\}$ であることと、$u,v$ 間に色 $x$ の辺が存在することが同値である。
したがって、$S:=\{f(e) \mid e\in E(G)\}$ に含まれない $N$ の元が存在することを示せば良い。
もし $G$ が多角形を含むならば、その多角形には必ず同じ色の辺が偶数本ずつ含まれる。そのため、多角形から適当な辺を1つ取り除くことで、$S$ を変化させることなく、$G$ に含まれる多角形の個数を1つ以上減らすことができる。この操作を繰り返すことで $G$ は最終的に森になるが、このとき $|S|\leq |E(G)|<|V(G)|=n=|N|$ となり、示せた。
