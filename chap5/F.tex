\subsection{}
各 $A_i$ から要素 $a_i\in A$ を一様に乱択する.
$(a_1,\ldots,a_n)$ が代表系となるよう確率が正であることを示せばよい.

$i < j$ に対して確率変数 $X_{ij}$ を,$a_i = a_j$ のときに $1$,そうでないとき $0$ として定める.
期待値は $E(X_{ij}) = P(a_i=a_j) = \frac{A_i\cap A_j|}{|A_i|\cdot|A_j|}$ である.

$X = \sum_{ij} X_{ij}$ とすると,$E(X) = \sum_{ij}E(X_{ij})$ であるが,$E(X_{ij})$ の計算と問題の仮定より,$E(X) < 1$ である.
したがって $P(X<1) > 0$ となるので,$X<1$ となる選び方 $(a_1,\ldots,a_n)$ が存在する.これは代表系である.