\subsection{}
一般に次のようにして,De Bruijn 系列を構成できる:

\begin{itembox}[l]{De Brujin 系列の構成}
 $q = 2^n$ とする.$\alpha$ を $\F_{q}$ の原始根の生成元とする.
 $0$ でない線形写像 $\phi\colon \F_{q} \longrightarrow \F_2$ をとる.
 $q-1$ 周期の $(0,1)$ 列 $\{a_i\}$ を,$a_i = \phi(\alpha^i)$ により定める.
 
 このとき,$a_i = a_{i+1} = \cdots = a_{i+n-2} = 0$ ($n-1$ 個並ぶ) 箇所が,周期内に唯一存在することが分かる.
 そこにひとつ $0$ を挿入したものが De Bruijn 系列を与える.
\end{itembox}

これを示す.

まず,$0$ を挿入する前の状態について考える.$q-1$ 個ある長さ $n$ の連続部分列が,すべて異なることを示そう.
$a_i$, $a_j$ から始まる長さ $n$ の列が完全一致するとして,$q-1\mid i-j$ を示せばよい.
このとき$0\leq k < n$ に対して $\alpha^{i+k} - \alpha^{j+k} \in \Ker \phi$ が成り立つ.
$\dim \Ker \phi = n-1$ なので,これらは線形従属.
$\sum_{0\leq k < n} c_k(\alpha^{i+k} - \alpha^{j+k}) = 0$ となる非自明な係数 $c_0, \ldots, c_{n-1}$ が存在.
$(\alpha^i-\alpha^j)\sum_{0\leq k < n}c_k\alpha^k = 0$ である.

$\alpha$ は $\F_q$ の生成元であるから,$n-1$ 次以下の非自明な多項式の根にならない.
したがって $\alpha^i - \alpha^j = 0$ が分かり,$\alpha$ を原始根であることから $q-1 \mid i-j$ となる.


以上により,相異なる長さ $n$ の連続部分列が $q-1$ 種類現れることが示された.
あとは,この中に列 $(0,0,\ldots, 0)$ が含まれないことを示せばよい.

$\alpha$ を根に持つ $n$ 次多項式 $x^n + \sum_{0\leq k < n} c_kx^k \in \F_2[x]$ をとる.
このとき,$\alpha^{i+n} + \sum_{0\leq k < n} c_k\alpha^{i+k} = 0$ であるから,数列 $\{a_n\}$ は
ある $n+1$ 項間斉次線形漸化式を満たす.よって,$(0,0,\ldots,0)$ ができるとそれ以降は永遠に
$a_i = 0$ になってしまうので矛盾する.


最後に,$\phi$ を $\sum_{0\leq k < n}c_k\alpha^k\longmapsto c_0$ とした場合が本問の構成である.