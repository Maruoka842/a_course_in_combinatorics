\subsection{}
$N:=2n+1$と置く。
$i-j=m\bmod N$のとき$c_{ij}=1
$、それ以外のとき$c_{ij}=0$と定める。
$\xi$を1の$N$乗根とする。
このとき
$$\sum_{j=0}^{N-1} c_{ij}\xi^{kj} = \xi^{mk} \xi^{kj}$$
であるから、$\bm{a}_k=^t(1,\xi^k,\xi^{2k},\ldots,\xi^{(N-1)k})$ $(k=0,1,\ldots,N-1)$は行列$C_m=(c_{ij})$の一次独立な固有ベクトルである。
頂点数$N$ (奇数)の多角形の距離行列は$\sum_{m=1}^n m(C_m+C_{-m})$と表されるから、$\bm{a}_k$は距離行列の一次独立な固有ベクトルでもある。$k\neq 0$のとき$\bm{a}_k$の固有値は

\begin{align*}
 \sum_{m=1}^{N-1} i(\xi^{km}+\xi^{-km})
=& \left.\frac{\partial}{\partial x} \frac{1}{1-x}\right|_{x=\xi^k}+\left.\frac{\partial}{\partial x} \frac{1}{1-x}\right|_{x=\xi^{-k}}\\
=& \frac{1}{(1-\xi^k)^2}+\frac{1}{(1-\xi^{-k})^2}\\
=& \frac{\xi^{-2k}+\xi^{2k}-2\xi^{-k}-2\xi^{k}}{(2-\xi^k-\xi^{-k})^2}\\
=& \frac{2\cos\left(2\pi \frac{2k}{N}\right)-4\cos\left(2\pi \frac{k}{N}\right)}{\left(2-2\cos\left(2\pi \frac{k}{N}\right)\right)^2}\\
\end{align*}
となる。$0<|\cos\left(2\pi \frac{m}{N}\right)|<1$より
\begin{align*}
2\cos\left(2\pi \frac{2m}{N}\right)-4\cos\left(2\pi \frac{m}{N}\right)
=&4\cos^2\left(2\pi \frac{m}{N}\right)-4\cos\left(2\pi \frac{m}{N}\right)-2\\
=&4\left(\cos\left(2\pi \frac{m}{N}\right)-\frac{1}{2}\right)^2-1\\
< &0\\
\end{align*}
であるから$a_{1},a_{2},\ldots,a_{N-1}$に対応する固有値は全て負である。一方$a_0$の固有値は$2\sum_{i=1}^n i$であるから正である。よって定理9.1と定理9.6より$N(P_{2n})=2n$である。
