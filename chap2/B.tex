\subsection{}
\prufer の校正における列 $(x,y) = ((x_1,ldots,x_{n-1}), (y_1,\ldots,y_{n-1})$ において,
各 $x_iy_i$ が木の辺に対応する.特に,頂点 $v$ の次数は $x_1,\ldots, x_{n-1}, y_1, \ldots, y_{n-1}$ における
$v$ の出現回数に等しい.

列 $x$ には $1,2,\ldots, n-1$ がちょうど一度ずつ現れる.よって,列 $y$ における $v$ の出現回数は $\deg(v) - 1$ に等しい.

以上の観察により,木 $T$ が本問の次数条件を満たすことは,その \prufer $P(T)$ において,
\begin{itemize}
 \item $2$ と $3$ が $2$ 度ずつ現れる.$5$ が $1$ 度現れる.他の数は現れない.
\end{itemize}
が成り立つことと同値である.木と \prufer の一対一対応より,木を数えることは,この条件を満たす数列の数え上げと等価で,
$\binom{5}{2,2,1} = 30$ 通りと計算できる.