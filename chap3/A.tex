\subsection{}
(1) 背理法による。$H$ が可分であると仮定し、関節点 $x$ を1つとる。
$H \setminus \{x\}$ は 2つ以上の連結成分に分かれるので、$H \setminus \{x\}$ の頂点集合の非空な分割 $A,B$ であって、$AB$ 間に辺がないようなものが取れる。
最小性の仮定から、$A,B$ それぞれからなる $H$ の誘導部分グラフは $d$ 彩色可能である。
そのような彩色を1つとり、必要なら $B$ の彩色をpermutateすることで、$\Gamma_H(x)$ を $d$ 色未満にできるので、$H$ を $d$ 彩色でき矛盾。

(2)背理法による。頂点集合のある分割 $X,Y (|Y| \geq 3)$ であって、$\Gamma(y)\cap X \neq \emptyset$ を満たすような $y\in Y$ が高々2点しかないものが存在する。
(1)よりちょうど2点であるとしてよい。その2点を $a,b$ とする。
頂点集合を$X\cup \{a,b\}, Y$とする誘導部分グラフに、それぞれ $(a,b)$ を辺として加えたグラフを考える。
このグラフは命題の仮定を満たす($a,b$ の次数が $d$ 以下であることが確かめられる)。また、頂点数が $H$ より真に小さいので $d$ 彩色可能である。
そのような彩色を1つとり、$a,b$ の彩色が整合するように $Y$ の彩色をpermutateしてから合体させることで、$H$ を $d$ 彩色でき矛盾。
