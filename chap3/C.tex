\subsection{}
$a = N(p-1,q,2)$, $b = N(p,q-1;2)$ とする.
$n = a+b-1$ として,$K_n$ の辺を赤または青で塗る.赤の $K_p$ または青の $K_q$ がとれることを示せばよい.

赤い辺からなる部分グラフを $R$ とする.$n$ は奇数であり,奇点は偶数個であるから,$\deg_R(v)$ が偶数であるような $v$ が存在する.
$\deg_R(v) \geq a$ であるとき,その $a$ 個から赤い $K_{p-1}$ または青い $K_{q}$ がとれる.前者の場合 $v$ と合わせて赤い $K_p$ が得られ,
後者の場合そのまま青い $K_q$ が得られているので,この場合には示された.
$\deg_R(v) < a$ であるとする.$a$ と $\deg_R(v)$ はともに偶数なので,$\deg_R(v) \leq a-2$ が成り立つ.

青い辺からなる部分グラフを $B$ とすると,$\deg_B(v) = n - 1 - \deg_R(v) \geq b$ が成り立つ.よってこの場合も $\deg_R(v) \geq	a$ の場合と同様である.
