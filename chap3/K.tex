\subsection{}
$G$ を $d$ 色で彩色する.最も少ない回数使われた色が $1$ であるとしてよい(特に,色 $1$ の頂点は $\frac{n}{d}$ 以下である).
各色 $1$ の頂点 $v$ に対して,まわりには色 $2,\ldots,d$ のいずれかの色の頂点が合計で $d$ 個以下ある.
$2(d-1) > d$ であるから,ある色は $v$ の隣接点に $1$ 度しか現れていない.そのような頂点を $w_v$,色を $c_v$ とする.

$G$ を次のように変形すれば $d-1$ 色での彩色が得られる:
各 $v$ に対して辺 $vw_v$ を除く.$v$ の色を $c_v$ に変更する.

色 $1$ の取り方によりこの過程で除かれる辺の数は $\frac{n}{d}$ 以下であるからよい.